
\section{ACTA DE PROPIEDAD INTELECTUAL}
\begin{center}
\vspace{0.7cm}
\textbf{{\Large ACTA DE PROPIEDAD INTELECTUAL}}

\end{center}
\begin{center}
    \textbf{UNIVERSIDAD DEL CAUCA}\\ 
    \textbf{FACULTAD DE INGENIERIA ELECTRÓNICA Y TELECOMUNICACIONES} \\ 
    \textbf{\\ACTA DE ACUERDO SOBRE LA PROPIEDAD INTELECTUAL DE LA PRÁCTICA DE PREGRADO }
\end{center}

En atención al acuerdo del Honorable Consejo Superior de la Universidad del Cauca, número 008 del 23 de Febrero de 1999, donde se estipula todo lo concerniente a la producción intelectual en la institución, los abajo firmantes, reunidos el día 10 del mes de Agosto de 2023, acordamos las siguientes condiciones para el desarrollo y posible usufructo del siguiente proyecto. \\ \\ 


Materia del acuerdo: Trabajo de grado para optar el título de \textbf{Ingeniero en Electrónica y Telecomunicaciones.\\ \\}

Título del Proyecto: \textbf{Desarrollo de un proceso replicable de seguridad en capas para aplicaciones web con Node.js: Identificación y mitigación de vulnerabilidades comunes.\\ \\} 

Objetivo del proyecto: \textbf{Definir un esquema/proceso replicable para la validación de pre-requisitos técnicos que minimicen la presencia de vulnerabilidades en plataformas desarrolladas en la empresa con Node.js. \\ \\}

Duración del proyecto: \textbf{6 meses \\ \\}

Cronograma de actividades: \textbf{Ver Sección del cronograma} \\ \\

Término de vinculación de cada partícipe en el mismo: \textbf{6 Meses} \\ \\

Organismo financiador: \textbf{TBBC (The BitBang Company)} \\ \\


Los participantes del proyecto, el estudiantes de pregrado \textbf{Andrés Felipe Diago Matta}, identificados con cédula de ciudadanía número 1.001.086.246 de Popayán, a quien en adelante se le llamará ''estudiante'', el ingeniero \textbf{Erwin Meza Vega} en calidad de Director del trabajo de grado, identificado con la cédula de ciudadanía 13.718.987 a quien en adelante se le llamará ''docente'', , y la compañía WIZIT MIND BLOWING SOLUTIONS S.A.S, representada por la ingeniera \textbf{Samara Catalina Enriquez Urbano}, a quien en adelante se le llamará ''organización'', manifiestan que: 

\begin{enumerate}
    \item La organización abrió su oferta para practicantes de desarrollo de software, la cual fue acogida por el estudiante como modalidad de grado práctica profesional.
    \item La idea fue comunicada del estudiante al docente, para realizar el proyecto de grado y desarrollar la práctica profesional bajo la dirección del docente. 
    \item Los derechos intelectuales y morales corresponden a la compañía.
    \item Los derechos patrimoniales corresponden a la compañía. 
    \item Los participantes se comprometen a cumplir con todas las condiciones de tiempo, recursos, infraestructura, dirección, asesoría, establecidas en el anteproyecto, a estudiar, analizar, documentar y hacer acta de cambios aprobados por el Consejo de Facultad, durante el desarrollo del proyecto, los cuales entran a formar parte de las condiciones generales.
    \item El docente y los estudiantes se comprometen a dar crédito a la Universidad y de hacer mención del Fondo de Fomento de Investigación en caso de existir, en los informes de avance y de resultados, y en registro de éstos, cuando ha habido financiación de la Universidad o del Fondo.
    \item Cuando por razones de incumplimiento, legalmente comprobadas, de las condiciones de desarrollo planteadas en el anteproyecto y sus modificaciones, el participante deba ser excluido del proyecto, los derechos aquí establecidos concluyen para él. Además se tendrán en cuenta los principios establecidos en el reglamento del programa y el acuerdo 035 de 1992 vigente de la Universidad del Cauca en lo concerniente a la cancelación y la pérdida del derecho a continuar estudios. 
    \item El documento del anteproyecto y las actas de modificaciones si las hubiere, forman parte integral de la presente acta.
    \item Los aspectos no contemplados en la presente acta serán definidos en los términos del acuerdo 008 del 23 de febrero de 1999 expedido por el Consejo Superior de la Universidad del Cauca, del cual los participantes del acuerdo aseguran tener pleno conocimiento. \\ \\ \\ \\
\end{enumerate}

\newpage

\begin{flushleft}
    \underline{\ \ \ \ \ \ \ \ \ \ \ \ \ \ \ \ \ \ \ \ \ \ \ \ \ \ \ \ \ \ \ \ \ \ \ \ \ \ \ \ \ \ \ \ \ \ }
\end{flushleft} 
Erwin Meza Vega \\ 
\textbf{Director \\ \\ } 


\begin{flushleft}
    \underline{\ \ \ \ \ \ \ \ \ \ \ \ \ \ \ \ \ \ \ \ \ \ \ \ \ \ \ \ \ \ \ \ \ \ \ \ \ \ \ \ \ \ \ \ \ \ }
\end{flushleft} 
Andrés Felipe Diago Matta \\ 
\textbf{Estudiante \\ \\} 


\begin{flushleft}
    \underline{\ \ \ \ \ \ \ \ \ \ \ \ \ \ \ \ \ \ \ \ \ \ \ \ \ \ \ \ \ \ \ \ \ \ \ \ \ \ \ \ \ \ \ \ \ \ }
\end{flushleft} 
Samara Catalina Enriquez Urbano \\ 
\textbf{Asesor en WIZIT MIND BLOWING SOLUTIONS S.A.S \\ \\} 


\begin{flushleft}
    \underline{\ \ \ \ \ \ \ \ \ \ \ \ \ \ \ \ \ \ \ \ \ \ \ \ \ \ \ \ \ \ \ \ \ \ \ \ \ \ \ \ \ \ \ \ \ \ }
\end{flushleft} 
Alejandro Toledo Tobar \\ 
\textbf{Decano \\ \\} 


\includepdf[pages={1,2,3}]{img/Acta de propiedad intelectual Firmada - Andres.pdf}
