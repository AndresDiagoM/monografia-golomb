\chapter[GENERALIDADES]{\Large GENERALIDADES}
\label{chap1:generalidades}
\justify %%justificar los párrafos

Una vez definido el motivo y los objetivos del trabajo y antes de dar un enfoque
directamente sobre el problema, es necesario saber la situación actual del mismo.
Comprender el objetivo, las tecnologías y arquitecturas utilizadas, es esencial para conocer el contexto del problema. En este capítulo hablaremos sobre los conceptos principales que se van a tratar a lo largo del proyecto.



\section{Four Wave Mixing (FWM)}
La Mezcla de Cuatro Ondas (FWM, por sus siglas en inglés \textit{Four Wave Mixing}) es un fenómeno no lineal que se manifiesta en sistemas de fibra óptica. Este efecto surge cuando dos o más señales ópticas, con frecuencias centrales diferentes, coexisten y se propagan a lo largo de una misma fibra.

En condiciones normales, estas señales ópticas deberían viajar independientemente sin interactuar significativamente entre sí. Sin embargo, debido a las no linealidades inherentes a la fibra óptica, como la no linealidad Kerr, las señales pueden influenciarse mutuamente.

El proceso de FWM implica la generación de nuevas componentes de frecuencia en la señal óptica original. Específicamente, se generan sumas y diferencias de las frecuencias originales, dando lugar a componentes espectrales adicionales. Este fenómeno puede introducir interferencias indeseadas entre los canales de comunicación en sistemas de multiplexación por división de longitud de onda (WDM), afectando negativamente la calidad de la señal y, en última instancia, la integridad de la transmisión de datos.

La FWM se convierte en un desafío significativo, especialmente en entornos donde se utilizan múltiples canales con frecuencias cercanas, como en sistemas WDM con canales igualmente espaciados. La supresión efectiva de la FWM es esencial para garantizar un rendimiento óptimo en las redes ópticas y maximizar la capacidad del sistema.

En el contexto de este trabajo, la propuesta de un mecanismo dinámico para contrarrestar la FWM en una red MLR-PON busca abordar este desafío, utilizando estrategias como la asignación de canales desigualmente espaciados y algoritmos optimizados para mitigar los efectos perjudiciales de la FWM.


\section{Redes MLR-PON}
Una Red de Acceso de Largo Alcance con Múltiples Longitudes de Onda (MLR-PON, por sus siglas en inglés \textit{Long-Reach Multiple Wavelength Passive Optical Network}) es una variante evolucionada de la \acrfull{pon}. A diferencia de las PON convencionales, las MLR-PON están específicamente diseñadas para proporcionar alcances extendidos, lo que las hace adecuadas para entornos donde se requiere cobertura en áreas geográficas más extensas.

Las redes MLR-PON logran este aumento en el alcance mediante el empleo de múltiples longitudes de onda en la transmisión de señales ópticas a través de la fibra. Cada longitud de onda puede considerarse como un canal independiente, lo que permite la transmisión simultánea de datos a diferentes velocidades y servicios.

Este enfoque de múltiples longitudes de onda no solo extiende el alcance de la red, sino que también proporciona una mayor flexibilidad y capacidad de ancho de banda. Además, las MLR-PON son compatibles con servicios de voz, datos y video, ofreciendo una solución integral para las demandas crecientes de conectividad en entornos residenciales y empresariales.

La implementación de MLR-PON se ha vuelto crucial para superar las limitaciones de distancia asociadas con las PON tradicionales, especialmente en áreas geográficas dispersas. En el contexto de este trabajo, la propuesta de un mecanismo dinámico para contrarrestar la Mezcla de Cuatro Ondas (FWM) se enfoca en optimizar el rendimiento de las MLR-PON, abordando los desafíos específicos asociados con la interferencia no lineal en este tipo de redes.


\section{Asignación de Canales Desigualmente Espaciados}
La asignación de canales desigualmente espaciados es una estrategia de gestión del espectro en sistemas de transmisión óptica, donde las longitudes de onda utilizadas para la transmisión no siguen una separación uniforme. En contraste con la asignación de canales igualmente espaciados, donde la distancia entre las longitudes de onda adyacentes es constante, la asignación desigual implica una distribución no uniforme de las longitudes de onda a lo largo del espectro óptico.

Esta técnica se utiliza para abordar problemas específicos relacionados con la interferencia y la diafonía en sistemas ópticos, especialmente en entornos de multiplexación por división de longitud de onda (WDM). La asignación desigual de canales puede optimizarse para minimizar la interferencia no lineal, como la Mezcla de Cuatro Ondas (FWM, por sus siglas en inglés \textit{Four Wave Mixing}), que puede ocurrir cuando las señales ópticas tienen intensidades significativas y se propagan simultáneamente en la fibra.

Esta estrategia permite adaptar la distribución de las longitudes de onda a las características específicas de la red óptica, teniendo en cuenta las propiedades de transmisión de la fibra y los efectos no lineales asociados. Al asignar desigualmente las longitudes de onda, se pueden evitar ciertos patrones que contribuyen a la interferencia no deseada, mejorando así el rendimiento del sistema.

En el contexto de este trabajo, la propuesta de un mecanismo dinámico para contrarrestar la FWM en una red MLR-PON se centra en la asignación de canales desigualmente espaciados como una estrategia clave para mitigar los efectos no lineales y mejorar la calidad de la transmisión en este tipo de arquitecturas.


\section{Reglas de Golomb}
Las reglas de Golomb y su variante generalizada, conocida como reglas g-Golomb, desempeñan un papel crucial en la asignación de canales en sistemas ópticos, especialmente en el contexto de la multiplexación por longitud de onda (WDM). Estas reglas se basan en conjuntos de enteros positivos que exhiben propiedades matemáticas particulares, lo que las hace extremadamente útiles en la resolución de problemas asociados con la interferencia de señales ópticas \cite{Apostolos}.

En su forma más fundamental, las reglas de Golomb son conjuntos de números enteros positivos en los que las diferencias entre los elementos son distintas y están distribuidas de manera uniforme. Este comportamiento único proporciona una propiedad valiosa para la asignación de frecuencias en sistemas WDM, donde la interferencia entre canales adyacentes es una preocupación significativa.

La generalización de estas reglas, conocida como reglas g-Golomb, extiende su aplicabilidad al permitir conjuntos de enteros más grandes con distancias que pueden repetirse un número limitado de veces. Esta flexibilidad es esencial al abordar problemas prácticos en la asignación de canales, ya que se pueden crear conjuntos más grandes y complejos, lo que resulta beneficioso para contrarrestar fenómenos no lineales como la mezcla de cuatro ondas (FWM).

Al utilizar reglas de Golomb o g-Golomb en la asignación de frecuencias, se logra una distribución estratégica de los canales que ayuda a mitigar la interferencia y optimizar el rendimiento del sistema. Además, al evitar la necesidad de búsquedas computacionales exhaustivas en conjuntos de mayor orden, como es el caso con las reglas g-Golomb, se facilita el diseño de algoritmos eficientes para la asignación de canales en sistemas ópticos avanzados.

En este contexto, explorar las propiedades y aplicaciones de las reglas de Golomb y su generalización se convierte en un componente esencial de la investigación y el desarrollo de soluciones para los desafíos asociados con la FWM en sistemas de comunicaciones ópticas.

\section{Algoritmos para Asignación de Canales}
Los algoritmos para la asignación de canales desempeñan un papel crucial en la eficiencia y el rendimiento de los sistemas de comunicaciones ópticas, especialmente en entornos donde la interferencia entre canales es una consideración crítica. Estos algoritmos son esenciales para optimizar la distribución de frecuencias en sistemas de multiplexación por longitud de onda (WDM), y su diseño adecuado puede influir significativamente en la reducción de fenómenos no lineales, como la mezcla de cuatro ondas (FWM).

En el corazón de la asignación de canales se encuentra la necesidad de evitar la interferencia entre las señales ópticas transmitidas simultáneamente en una fibra. Los algoritmos dedicados a esta tarea buscan una distribución espacial y espectral eficiente de los canales, maximizando el ancho de banda disponible y minimizando la diafonía.

Estos algoritmos pueden clasificarse en diversas categorías, que incluyen métodos de asignación uniforme y métodos de asignación desigualmente espaciados. Mientras que los métodos de asignación uniforme buscan una separación constante entre las frecuencias de los canales, los métodos desigualmente espaciados, como los basados en reglas de Golomb o g-Golomb, buscan una separación no uniforme para minimizar la interferencia.

La complejidad de estos algoritmos varía según el enfoque y la naturaleza específica del sistema en cuestión. Algunos algoritmos se centran en la búsqueda exhaustiva de conjuntos óptimos, mientras que otros pueden aprovechar enfoques heurísticos para ofrecer soluciones rápidas y eficientes.

En el contexto de esta investigación, se explorarán y evaluarán algoritmos específicos, incluidos aquellos que aprovechan las propiedades de las reglas g-Golomb para la asignación de canales desigualmente espaciados. La ventaja de estos algoritmos radica en su capacidad para abordar problemas de interferencia y diafonía, proporcionando soluciones óptimas o aproximadas que mejoran el rendimiento general del sistema.

La investigación y desarrollo de algoritmos para la asignación de canales se convierte así en un aspecto esencial para abordar los desafíos asociados con la FWM en redes de multiplexación por longitud de onda pasiva reconfigurable (MLR-PON).

\section{Monitoreo de Desempeño Óptico - OPM}

El monitoreo de desempeño óptico (OPM) representa una faceta crítica en la gestión de redes de comunicación óptica, desempeñando un papel clave en la evaluación y aseguramiento de la calidad de la transmisión de señales ópticas. Esta técnica proporciona una visión detallada de diversos parámetros ópticos, permitiendo una supervisión en tiempo real y una respuesta proactiva a posibles problemas que puedan afectar el rendimiento del sistema.

En el contexto de la tesis, el OPM se erige como una herramienta fundamental para evaluar la integridad de las señales ópticas en una red MLR-PON, donde la interferencia y los fenómenos no lineales, como la mezcla de cuatro ondas (FWM), pueden tener un impacto significativo.

El monitoreo de desempeño óptico implica la medición de diversas magnitudes, como la potencia óptica, la atenuación y la calidad de la señal. Estos datos son esenciales para evaluar la salud general del sistema y para anticipar cualquier degradación del rendimiento antes de que afecte la calidad de la transmisión.

La integración de herramientas de OPM en una red MLR-PON permite la identificación temprana de posibles puntos problemáticos, facilitando así la implementación de medidas correctivas. Esto resulta crucial en entornos dinámicos, donde las condiciones de la red pueden cambiar debido a factores ambientales, mantenimiento, o variaciones en la demanda del servicio.
