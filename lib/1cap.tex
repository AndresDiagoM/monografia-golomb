\setcounter{page}{1}

\chapter{Introduction}
\label{chap1:introduction}

\begin{center}
    \item \section{PLANTEAMIENTO DEL PROBLEMA}
\end{center}


\justify %%justificar los párrafos

\vfill %espaciar los párrafos

%---revisar articulo 20, en lista primera búsqueda

En los últimos años ha tenido lugar un crecimiento del uso de las \gls{tic}. Esto ha provocado un gran incremento en el interés de las empresas por estar online y prestar sus servicios a través de plataformas digitales, actualizando sus herramientas de trabajo y digitalizando al máximo sus medios de producción. Este proceso de digitalización se ha visto potenciado por la proliferación de los servicios de almacenamiento y gestión en la nube, así como por la proliferación de dispositivos móviles inteligentes de bajo coste \cite{PedroRuiz}. 

Por esta razón, es importante abordar el tema de la seguridad, ya que la sociedad actual usa un gran número de aplicaciones, desde la banca en línea y aplicaciones de trabajo remoto hasta el entretenimiento personal y el comercio electrónico. Es por esto que, las aplicaciones se convierten el objetivo principal de los atacantes, que se aprovechan de las vulnerabilidades como los fallos de diseño, así como de las debilidades de las API, el código abierto, los widgets de terceros y el control de acceso, buscando acceder a bases de datos, servidores y otros datos sensibles. Si se logra la exposición de datos sensibles, es posible lanzar ataques de referencia u otras formas de fraude en línea.

Los atacantes digitales tienen la ventaja de contar con el  alcance y anonimato de quien los propicia, por lo que significa un gran reto para cualquier organización. Como prácticas de seguridad habituales se tiene el cambio de contraseñas, prohibir acceso a dispositivos y hacer una constante actualización de software. Pero esto no implica que la seguridad de las aplicaciones sea un factor que siempre se considere, siendo así un elemento por lo general ignorado y vulnerable. Esto genera una alta probabilidad de enfrentarse a amenazas que se generan por fallos del sistema a raíz de una inadecuada codificación, servidores mal configurados y un mal diseño en la aplicación.


Este trabajo de grado se va a desarrollar para la empresa WIZIT MIND BLOWING SOLUTIONS S.A.S, la cual es una empresa que se especializa en crear soluciones tecnológicas a la medida para sus clientes. El portafolio de servicios que ofrece, se encuentran soluciones en desarrollo de software, ciencia de datos, servicios en la nube, consultoría digital, entre otras líneas de negocio. Dentro de las soluciones que tienen una relación con plataformas tecnológicas, digitalización de procesos y software que está en la web, se utiliza entre otras tecnologías Node.js, que es una plataforma de desarrollo muy popular para aplicaciones web, y con su creciente adopción, la seguridad de las aplicaciones desarrolladas en Node.js se ha convertido en una preocupación crítica. 


Con lo mencionado anteriormente, este trabajo tiene como propósito la definición de un esquema/proceso replicable para la validación de pre-requisitos técnicos puede ayudar a minimizar la presencia de vulnerabilidades en estas aplicaciones, permitiendo un análisis de las vulnerabilidades comunes en aplicaciones desarrolladas con Node.js y la identificación de los pre-requisitos técnicos necesarios para prevenirlas. También implica la definición de un proceso replicable para validar la presencia de estos pre-requisitos técnicos en las plataformas de la empresa.



%%norma iso27001 -> https://www.iso.org/standard/27001
