\chapter[MARCO METODOLÓGICO]{\Large MARCO METODOLÓGICO}
\label{chap2:marco}

En este capítulo se presentan las herramientas y procedimientos abordados en la
realización del trabajo de grado. Así mismo, se describe la herramienta de software
escogida para el diseño y la simulación de la red. Luego, se presenta la metodología seleccionada y su aplicación dentro del trabajo de grado.

\section{Herramienta de Simulación}

Es esencial llevar a cabo una simulación o modelado exhaustivo de un sistema de comunicaciones antes de su implementación real. En este sentido, resulta fundamental escoger una herramienta o software de simulación que permita monitorear y analizar eficazmente el comportamiento de la señal a través del canal óptico deseado. La herramienta debe posibilitar la recreación del funcionamiento de la red en un entorno lo más cercano posible a la realidad, facilitando la evaluación del rendimiento mediante la consideración de los parámetros de Monitoreo de Desempeño Óptico (OPM). 
En la presente sección, se presenta y describe la herramienta de simulación. Así mismo, se analiza su comportamiento, la disponibilidad de licencia (en caso de ser aplicable) y la capacidad de implementar tecnologías a nivel de enlace y de red. A continuación, la descripción de la herramienta a considerar:

\subsection{OptSim}

OptSim es una herramienta desarrollada por Synopsys, basada en un entorno de simulación y modelado intuitivo destinado al diseño y la evaluación del rendimiento en el nivel de transmisión de sistemas de comunicaciones ópticas. Esta herramienta dispone de una amplia librería que incluye los componentes más comúnmente utilizados en sistemas de comunicaciones ópticas, clasificados en diversas categorías como transmisores, generadores de señales, fibras ópticas, multiplexores, demultiplexores, receptores, entre otros.

Esta estructura permite lograr un equilibrio óptimo entre precisión y eficiencia temporal, facilitando la optimización de los diseños para mejorar el rendimiento y reducir costos. OptSim se distingue por su elevado nivel de eficiencia y precisión, mientras que su interfaz gráfica es de fácil manipulación, lo que contribuye a una experiencia de usuario intuitiva y eficaz.