\newpage
\section{Anexos}
\subsection{Documentos relacionados con comparación de imágenes}
    \begin{itemize}
    \item El artículo \textbf{Comparison of Geo-Object Based and Pixel-Based Change Detection of Riparian Environments using High Spatial Resolution Multi-Spectral Imagery}  \cite{article23} describe el desarrollo de un sistema de clasificación basado en objetos geográficos para mapear con precisión las clases de cobertura terrestre ribereña para dos imágenes y comparar mapas de cambio derivados de mapas basados en objetos geográficos y por píxel.  Por lo que las técnicas de comparación se realizan de dos diferentes maneras, primero con objetos geográficos como entradas y luego por píxel. Las técnicas de detección de cambios incluyeron la comparación posterior a la clasificación, la diferenciación de imágenes y la transformación Tasseled Cap. Del estudio se concluye que las entradas basadas en objetos geográficos proporcionaron resultados de detección de cambios más precisos que los derivados de las entradas basadas en píxeles, ya que el enfoque basado en objetos geográficos redujo los efectos de sombras y registros erróneos y permitió la inclusión de relaciones de contexto.
    Este documento aporta a nuestro proyecto ya que evidencia la necesidad de una clasificación antes del proceso de comparación porque al clasificar los objetos urbanísticos como las edificaciones se generan unos resultados de detección de cambios más precisos, que si se hace una comparación por píxeles.
    Nuestro trabajo se diferencia de éste, debido a que en este trabajo se quiere analizar los cambios de vegetación a diferencia de nuestro proyecto en donde se requiere analizar cambios urbanos. Adicional a eso se hace el proceso con imágenes Quickbird, las cuales son imágenes satelitales, debido a que no se pueden implementar en el caso de nuestro proyecto ya que hay una limitación tecnológica porque no se disponen de satélites como este en la zona que se quiere analizar, entonces se plantea el proceso con imágenes aéreas provenientes de un dron.
    
    \item El artículo \textbf{Desarrollo de una herramienta de análisis de modificaciones en el terreno mediante el uso de imágenes de drones georreferenciadas o satelitales}  \cite{prostychenko_automatizacion_nodate} describe el desarrollo de un sistema que permite realizar actividades de forma automática correspondientes al proceso de seguimiento de la construcción de infraestructuras de parques fotovoltaicos, implementando para ello herramientas  software como Matlab y LabVIEW que facilitan el seguimiento de la evolución gracias al procesamiento digital de ortofotos geolocalizadas obtenidas mediante drones o alternativamente por satélite, además de la automatización de todo el proceso: adquisición de fotografías y su procesamiento, detección de los distintos elementos del terreno como paneles solares instalados, comparación con resultados anteriores y planos técnicos, obtención de resultados de avance y generación de informes.
    Este documento aporta a nuestro proyecto en la implementación y desarrollo de herramientas que permitan la realización de un proceso completo de detección de modificaciones entre imágenes de manera automática, desde la adquisición de fotografías y su procesamiento, detección de los distintos elementos del terreno, además de comparación con resultados anteriores y planos técnicos, que permiten obtención de resultados de avance y generación de informes. Por otra parte, se centran en el análisis y seguimiento de la construcción de infraestructuras de parques fotovoltaicos, alejándose de nuestro centro de estudio que correspondiente a la zona de construcción urbana, además de tomar en consideración análisis con imágenes satelitales, las cuales no están previstas dentro de nuestro proyecto por factores de tiempo y costos.
    \item El artículo \textbf{Monitoring of urban growth in the state of Hidalgo using Landsat images}\cite{LeautaudValenzuela2017} Se realiza el análisis de una técnica para la detección de plaga forestal mediante fotografías aéreas infrarrojas.  La utilización de fotografías aéreas digitales en color e infrarrojo permitió obtener imágenes VIR (visible + infrarrojo) con 4 bandas y una resolución aproximada de un metro por píxel. Las principales repercusiones en las áreas a evaluar tienen relación con las políticas de desempeño forestal, que no siguen el funcionamiento (incluido el saneamiento) en el área. Las fotografías aéreas digitales son útiles para la detección de árboles afectados en los bosques por medio de la interpretación visual con una eficiencia del 98 por ciento. 
    El proyecto contribuye con la demostración de técnica de detección de daños en el área seleccionada, que podría ser adaptado en este caso para cambios en estructuras, esta detección se realiza mediante el análisis de imágenes con uso de tecnología infrarroja, que corresponde a un aporte a resaltar por la seguridad en los resultados y una facilidad relativa en el proceso de revisión de los mismos que puede ser un proceso extenso con imágenes satelitales.
    Se difiere principalmente en las técnicas propuestas a usar para la comparación de las imágenes, el adaptar la tecnología de detección descrita en el artículo implicaría un cambio significativo en el proceso que se desea llevar con el análisis de las imágenes tomadas por drones, teniendo en cuenta que las plataformas software que ofrecen imágenes satelitales no reciben una actualización constante en el área a la que esta dirigida el proyecto, obstaculizando los periodos de tiempo deseados para la comparación de imágenes.
    \end{itemize}
\subsection{Documentos relacionados con control urbano}
 \begin{itemize}
        \item El artículo \textbf{Satellite monitoring of urbanization and environmental impacts—A comparison of Stockholm and Shanghai }\cite{HAAS2015138} plantea el monitoreo del crecimiento urbano a partir de imágenes satelitales, con el fin de ver el impacto que tiene el cambio de uso de la tierra en el medio ambiente, para ello se usan técnicas de análisis de imágenes como el análisis de texturas mediante la matriz de co-ocurrencia de nivel gris (GLCM) el cual mide las variaciones locales en la matriz de coocurrencia de nivel gris, esto con el fin de generar una clasificación de los mapas de uso del suelo y cobertura del suelo (LULC), posteriormente se usa una Máquina de Vector Soporte para clasificar las diferentes zonas entre agricultura, bosque, agua (humedales y acuicultura).
        Este artículo aporta a nuestro proyecto el uso de la Máquina Vector ya que este método puede clasificar las edificaciones de otros elementos del paisaje urbano como carreteras, parques entre otros, que son zonas que no se desean procesar para realizar la comparación lo que haría un sistema más eficiente ya que no se comparan elementos innecesarios.
        La principal diferencia entre este trabajo y el nuestro es que no se genera una comparación píxel a píxel que permita la identificación exacta de los píxeles que cambian con respecto a la imagen anterior, o una comparación por objetos obtenido de la clasificación previa, sino que se genera una comparación global de las zonas duras actuales. En nuestro trabajo se requiere hacer una comparación píxel a píxel o por objetos para poder identificar la ubicación de la edificación que está presentando novedades.
        
        \item El artículo \textbf{A systematic review and assessment of algorithms to detect, characterize, and monitor urban land change}\cite{REBA2020111739} hace una revisión y evaluación sistemática de algoritmos para detectar, caracterizar y monitorear el cambio de suelo urbano, para dar a conocer la gran variedad de métodos disponibles en este campo.  En este artículo se especifica claramente las etapas por las que pasan las imágenes para generar la detección de cambios urbanos generales, las cuales son: el preprocesamiento de imágenes, la comparación de imágenes, que normalmente ocurre usando un operador de razón, razón logarítmica o razón basada en vecindad y el umbral del indicador de cambio de imagen.
        
        Este artículo aporta a nuestro proyecto una visión general de las técnicas usadas en diferentes estudios para el monitoreo urbano, organizándolo en diferentes etapas, de las cuales se resalta las de control los cuales se refieren al cambio de cobertura terrestre, en donde se identifica que el método de transformación mencionado se puede adaptar a la comparación que se necesita en el trabajo propuesto ya que este método se basa en el cálculo de vectores de diferencia entre unidades de análisis dadas tanto la magnitud como la dirección del cambio.
        Nuestro trabajo se diferencia de los documentos vistos en la revisión sistemática plasmada en este documento ya que, se necesita una comparación de cada uno de los puntos de la imagen y no una comparación general que es lo que se hace para ver el impacto de la urbanización con los años, además, en este documento no se considera el procesamiento previo de imágenes ni los pasos de evaluación de la precisión, procesos indispensables para realizar el procesamiento de las imágenes. Por otro lado, se plantea que los estudios realizados en el monitoreo de los cambios urbanos se realizan principalmente en países desarrollados que tienen ingresos altos como China y Estados Unidos por lo que se encuentra una brecha en países como Colombia. Finalmente se destaca la necesidad de más estudios que distinguen la variación intraurbana ya que los estudios actuales se limitan a ver el crecimiento urbanístico, pero no lo que está sucediendo al interior de las ciudades, factor que se quiere abarcar con nuestro trabajo.
        
        \item El artículo \textbf{Safety Challenges for Integrating U-Space in Urban Environments}\cite{9476883} presenta el proyecto desarrolado en Europa de la iniciativa U-space para permitir la integración de miles de drones en los cielos . El análisis concluye que el marco actual no es lo suficientemente maduro para abordar algunos de los desafíos emergentes identificados. El trabajo futuro debería incluir el uso de técnicas de ingeniería de resiliencia para mitigar los riesgos vinculados a los sistemas autónomos y un uso equilibrado de la Inteligencia Artificial, cuyos beneficios a medio / largo plazo superará los retos iniciales de seguridad.
        En el artículo se pueden conocer y relacionar los retos que implican el uso de red de drones para la obtención de imágenes satelitales en espacios urbanos, con técnicas presentadas en el marco regulador de la gestión del tránsito aéreo, dichos desafíos pueden ser tenidos en cuenta en el momento de diseño de recorrido y distribución de los drones que se pretenden implementar en la parte de hardware. Se propone técnicas de ingeniería de resiliencia para buscar la solución estos, con uso de sistemas autónomos acoplados con Inteligencia Artificial.
        El proyecto se centra en las implicaciones y procesos necesarios en el control de despliegue de una red de drones en áreas urbanas, sin tomar plantear un sistema que permita analizar las diferencias en imágenes obtenidas por los drones para tener en cuenta esos cambios interurbanos.
        
        \item El artículo \textbf{Análisis de la minería mediante la generación de cartografía
        por medio de drones (UAV) como implementación de análisis de riesgos en el Municipio de Une Cundinamarca}\cite{rojas_velasquez_alisis_nodate} propone un sistema de actualización de información cartográfico (POT, PBOT o EOT) a partir del uso de nuevas tecnologías como los drones para la toma de imágenes puesto que, permiten reducir el gasto de tiempo en este tipo de estudios, además de facilitar la accesibilidad a ciertos tipo de terrenos. El estudio que aquí se abarca se enfoca principalmente en las áreas de explotación minera para definir los componentes de riesgo que conlleva el desarrollo de esta actividad en el terreno, su desarrollo se enfoca en la toma de imágenes aéreas y su respectivo procesamiento.
        Aporta a nuestro proyecto desde la exposición de la falta de control en el esquema y plan de ordenamiento territorial del país y como este se ha ido deteriorando con el paso del tiempo, proponiendo una solución basada en la inclusión de tecnología que representan los UAV en temas relacionados a cartografía que permitan contribuir a actualizaciones de los instrumentos técnicos de urbanización como el POT, PBOT o EOT de los municipios. 
        A diferencia del nuestro, en este documento el centro de estudio está en los municipios de la zona rural ya que son las zonas de riesgo afectadas por el área de reserva minera, implicando brechas en cuanto a los resultados esperados en comparación a nuestro proyecto ya que, las características del suelo y parámetros de comparación a tener en cuenta en el procesamiento de imágenes resultan diferentes a los de la zona urbana en este caso el centro histórico de la ciudad de Popayán en donde se desea desarrollar nuestro proyecto.
        
        \item El artículo \textbf{Uso de la tecnología UAV en el marco de un proyecto urbanístico de escala media con fines de ordenamiento territorial} \cite{garcilar_lallana_uso_2019} plantea una comparación de la eficiencia de los métodos tradicionales y la implementación de nuevas tecnologías para la regulación del control urbano (ordenamiento territorial), a partir de la comparación entre terrenos, específicamente entre la superficie y perímetro obtenida a partir de un relevamiento de la zona y  el plano antecedente del mismo, a través de vuelos fotogramétricos con UAV y la utilización de subproductos, validando las condiciones necesarias que deberían presentar los terrenos y alrededores con el fin de definir la  mensura urbana del mismo.
        Este documento aporta a nuestro proyecto parámetros para tener en consideración durante la etapa de comparación de terrenos, con fines de control territorial,  ya que se toma en cuenta factores como la superficie y perímetro de los mismos, a partir, de un relevamiento realizado en la zona a través de vuelos fotogramétricos con UAV, adicionalmente nos da a conocer algunas retos a los que nos tendremos que enfrentar en cuanto a la información que nos brinde la institución pública en este caso la curaduría urbana, ya que algunos predios pueden no presentar planos antecedentes y por ende no se pueda hacer posible la comparación con el relevamiento realizado.
        A diferencia del nuestro, el proyecto centra gran parte de su desarrollo al control y procesamiento posicional del vuelo fotogramétrico, el cual no cumple con el interés de nuestro proyecto, adicional a esto, no se abarca desarrollo ni implementación de algún prototipo tecnológico, ya que todas las herramientas utilizadas son softwares existentes en el mercado. De igual manera, no se exponen acciones de notificación ante los predios que presentaron cambios en contra de la norma establecida.
        
        \item El artículo \textbf{Using unmanned aerial systems (UAS) for remotely sensing physical disorder in neighborhoods }\cite{grubesic_using_2018}  realiza un analisis del entorno ecologico de una localidad a través de la recolección de imágenes captadas por drones que permiten identificar detalles del desorden urbano para lograr analizar el comportamiento de los habitantes y el uso de espacios públicos y como estos factores inciden en la salud, criminalidad o cultura del lugar.
        Aporta a nuestro proyecto desde el análisis sobre las ventajas que tienen los UAS frente a otros métodos de recolección de información a través de imágenes en cuestión de calidad y detalles que se pueden obtener. aportando una visión general de las ventajas que nos puede traer el uso de drones. 
        El trabajo realizado en esta investigación se enfocaba en estudiar en las imágenes recolectadas el desorden físico evidenciado en terrenos o edificios abandonados, grafitis, lugares donde la gente deposita su basura, entre otros, con el objetivo de realizar un análisis del contexto del área analizada y como este puede influir en la salud de los habitantes, en como disfrutan los espacios públicos y en la criminalidad de la zona, alejándose del objetivo de nuestro proyecto.
        \end{itemize}
\subsection{Documentos relacionados con toma de imágenes aéreas}
 \begin{itemize}
        \item El artículo \textbf{Improvement of Workflow for Topographic Surveys in Long Highwalls of Open Pit Mines with an Unmanned Aerial Vehicle and Structure from Motion}  \cite{rs13173353} plantea la realización de levantamientos topográficos en minas activas es un desafío debido a las operaciones continuas y los peligros en muros altos. Se realizan vuelos con drones en busca de la mejor configuración en términos de modo de altitud y ángulo de cámara, para producir una representación digital de topografías utilizando Structure from Motion. El estudio se llevó a cabo en una mina de caolín activa cerca del Parque Natural del Alto Tajo del Centro-Este de España. Se demostró que un modo de vuelo de dron de fachada combinado con un ángulo de cámara nadir, y programado automáticamente con un software de planificación de misiones basado en computadora, proporciona las topografías más precisas y detalladas, en el menor tiempo y con mayor seguridad de vuelo. Estas topografías permitieron la detección de un 93,5 por ciento más de discontinuidades que los levantamientos por encima del nivel medio del mar, el enfoque común utilizado en las áreas mineras. 
        
        Es posible adaptar del artículo, cuya base es el uso de drones con la tecnología Structure from Motion para identificar falencias en áreas específicas, las técnicas descritas. Se tiene una descripción del acomodamiento de parámetros de la cámara de los drones, posición y ángulo para la obtención de imágenes óptimas para un análisis mejor de las mismas. Se diferencia ya que el objetivo de análisis de área que se realiza a partir de las fotografías obtenidas por los drones difiere al proceso que se quiere ampliar en el anteproyecto, pues este tiene relación con el cambio de estructuras en un área urbana en este anteproyecto, por lo que no se hace un énfasis en características físicas del área, sino en cambios que buscan ser detectados. Además del uso de una red amplia de drones, que en este caso no es necesaria pues se quiere abarcar como espacio el centro de Popayán, lo que no implicaría el uso de numerosos drones.
        
        \item El artículo \textbf{Generación de cartografía catastral urbana comparativa, mediante la utilización de geo-tecnologías con fines de ordenamiento territorial en la ciudad de Bolívar, cantón Bolívar, provincia del  Carchi} \cite{carrera_herrera_generacion_2018}  plantea el uso de vehículos no tripulados y GPS diferencial para realizar imágenes aéreas de la ciudad Bolívar con el fin de proponer un plan de ordenamiento territorial para la zona, entendiendo la dinámica de crecimiento que tiene actualmente la ciudad. Además, se realiza un análisis de cobertura de servicios y el uso del suelo actual.
        
        Este artículo aporta a nuestro proyecto una visión general de las En este articulo aporta a nuestro mediante la descripción del método utilizado para realizar el vuelo fotogramétrico digital realizado a una escala de 1:100 con GSD 6cm. Además, describe se describen los errores de exactitud que puede ser por omisión, consistencia lógica o exactitud temática respecto a la representación gráfica de los elementos a cartografiar.
        A pesar de utilizar el mismo medio para obtener las imágenes aéreas del territorio, en este caso la metodología se basa en la toma de imágenes, abstracción de datos, comparación con la referencia existente y proposición de modelos de orden territorial que permitan un buen crecimiento de la ciudad, por ende, no existe más de una toma de datos o la necesidad de una amplia base de datos para almacenar la información. Por otra parte, en este proyecto se basa en una zona mayormente agraria y para la toma de datos bastaba con captar la utilización del suelo, pero no detalles de las edificaciones como se necesita en nuestro proyecto.
        
        \item El artículo \textbf{Aplicación de fotogrametría con dron para la actualización de los factores físicos del catastro urbano del distrito de Ticapampa } \cite{morales_alvarado_aplicacion_2021}  muestra los resultados de una investigación realizada acerca del uso de fotogrametría con dron para actualizar los factores físicos del catastro en Ticapampa sustentar la planificación urbana y de servicios.
       Este artículo aporta a nuestro proyecto ya que durante el desarrollo del trabajo se describen todas las características y especificaciones utilizadas para la recolección de imágenes por medio de drones, enfocándose en el reconocimiento de características físicas generales de un área geográfica especifica, asemejándose en ese aspecto al objetivo de nuestro proyecto. 
        El objetivo de este proyecto se basa en obtener una actualización de los datos territoriales que no han sido actualizados por más de 50 años en el territorio seleccionado, por ende, no es necesario realizar más de una toma de imágenes como se plantea en nuestro proyecto para poder comparar datos actualizados que permitan la notificación de cambios recientes y un control de nuevas construcciones
        \end{itemize}
