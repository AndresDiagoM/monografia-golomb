\begin{center}
    {\textbf{Propuesta de un mecanismo dinámico que permita contrarrestar el efecto de FWM en una red MLR-PON }}\\
    \vfill

    \includegraphics[scale=0.7]{img/logo-unicauca.png} \\
    \vfill

    \textit{Monografía para optar por el título de Ingeniería Electrónica y Telecomunicaciones}\\
    Modalidad Trabajo de Investigación\\
    \vfill

    Santiago Sánchez Echavarría\\
    Andrés Felipe Diago Matta\\
    
    \vfill
    \textbf{Tutores:}\\
    Phd. Jose Giovanny López Perafán \\
    Phd. Carlos Andres Martos Ojeda \\
    
    \vfill
    Universidad del Cauca\\
    Facultad de Ingeniería en Electrónica y Telecomunicaciones\\
    Programa de Ingeniería en Electrónica y Telecomunicaciones\\
    Grupo de investigación de Nuevas Tecnologías en Telecomunicaciones - GNTT\\
    %Popayán, Cauca\\
    %Noviembre de 2023\\
    \date{\today}
    Popayán, \today
\end{center} 
