\begin{center}
    \item  \section{OBJETIVOS}
\end{center}

\subsection{Objetivo general}

Definir un esquema/proceso replicable para la validación de pre-requisitos técnicos que minimicen la presencia de vulnerabilidades en plataformas desarrolladas en la empresa WIZIT MIND BLOWING SOLUTIONS S.A.S con Node.js.

\subsection{Objetivos específicos}
\begin{itemize}
    \item Caracterizar y priorizar posibles vulnerabilidades que existen al desarrollar aplicaciones nodejs en conjunto con express.
    \item Generar recomendaciones para prevenir ataques a partir de vulnerabilidades en situaciones específicas\footnote{Manual de reglas/ checklist, en concordancia con la norma ISO 27001}.
    \item Desarrollar un piloto que aplique las reglas definidas.
    \item Evaluar la efectividad del esquema/proceso replicable propuesto en la prevención de vulnerabilidades en plataformas desarrolladas en la empresa con Node.js, a través de la aplicación piloto y la comparación de los resultados obtenidos antes y después de aplicar las reglas definidas.
\end{itemize}

%% RECOMENDACIONES: se recomienda incluir el aspecto de la evaluación de las reglas propuestas en uno de los objetivos
%% Buscar un propósito para el modelo que se va a definir, el objetivo que recomendaron añadir

%%PROPOSITO: es establecer un esquema/proceso replicable para la validación de pre-requisitos técnicos en el desarrollo de aplicaciones en la empresa con Node.js. Este modelo tiene como objetivo minimizar la presencia de vulnerabilidades en las plataformas desarrolladas, brindando un enfoque estructurado y sistemático para identificar y mitigar posibles vulnerabilidades.
%El modelo propuesto busca proporcionar pautas claras y específicas para el desarrollo seguro de aplicaciones con Node.js, teniendo en cuenta las mejores prácticas y consideraciones de seguridad. Además, busca caracterizar y priorizar las vulnerabilidades comunes asociadas al uso de Node.js y su integración con Express, permitiendo la generación de recomendaciones y reglas para prevenir ataques en situaciones específicas.

%% ¿como aplicar el manual o checklist?, es posible pre, o post a la entrega de un desarrollo, o en el proceso de desarrollo de un desarrollo
%% seria necesario familiarizar el manual o checklist para el equipo de desarrollo, para que se pueda aplicar 