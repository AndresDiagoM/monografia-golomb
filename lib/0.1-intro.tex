\setcounter{page}{1}

\addcontentsline{toc}{chapter}{INTRODUCCIÓN}
\chapter*{INTRODUCCIÓN}
\markboth{INTRODUCCIÓN}{}
\label{chap0:introducion}
\justify %%justificar los párrafos

\hspace*{1em} El continuo avance tecnológico en el ámbito de las comunicaciones ópticas ha posibilitado la transmisión eficiente de vastas cantidades de información a través de redes de fibra óptica. Aunque estas redes presentan numerosas ventajas, su despliegue no está exento de desafíos, especialmente derivados de fenómenos no lineales. La interacción entre la luz y el material de la fibra óptica da lugar a efectos no lineales, destacando su relevancia en sistemas de \acrfull{wdm}, donde múltiples canales de diferentes longitudes de onda coexisten con mínima separación. La alta intensidad óptica, consecuencia de la propagación simultánea de canales de alta potencia, puede desencadenar fenómenos críticos que impactan la eficacia de la comunicación óptica. Entre estos fenómenos no lineales, destaca la \acrfull{fwm}, un efecto óptico no lineal de tercer orden. La \acrshort{fwm} surge cuando dos o más señales ópticas con distintas frecuencias centrales se propagan en una misma fibra, generando armónicos que presentan frecuencias equivalentes a la suma o diferencia de las ondas originales.


En la búsqueda de soluciones para mitigar estos efectos no lineales, se han propuesto diversos enfoques, entre los cuales destacan los algoritmos de \acrfull{USCA}, los cuales han demostrado ser prometedores en la reducción de diafonía generada por señales co-propagantes. Para contrarrestar específicamente el impacto de la \acrshort{fwm}, se torna esencial una asignación eficiente de canales en las fibras ópticas. En este contexto, las reglas de Golomb, originadas en la teoría de números y caracterizadas por definir conjuntos de enteros con la particularidad de tener todas sus sumas o diferencias distintas, emergen como una herramienta valiosa. Al garantizar distancias únicas entre las frecuencias asignadas, se puede mitigar el efecto de la \acrshort{fwm} y mejorar la calidad de la transmisión.


Sin embargo, el empleo de reglas de Golomb para la asignación de canales no es una tarea trivial. El proceso de búsqueda y definición de estas reglas se vuelve computacionalmente desafiante, especialmente al considerar conjuntos de mayor tamaño.
Ante este panorama, surge la necesidad imperante de proponer un mecanismo dinámico que, aprovechando las propiedades matemáticas de las reglas de Golomb, permita contrarrestar eficientemente el efecto de la \acrshort{fwm} en las \acrfull{MLR-PON}, adaptándose a diversas condiciones y requisitos de la red. En el contexto de canales WDM uniformemente espaciados, la generación de términos \acrshort{FWM} en frecuencias adyacentes provoca diafonía entre canales, afectando el rendimiento general del sistema. La supresión de la \acrshort{fwm}, mediante la aplicación de reglas de Golomb, puede tener implicaciones prácticas significativas, como la posibilidad de utilizar canales más densamente espaciados en el espectro óptico, aumentando así la capacidad del sistema. Además, la atenuación de la \acrshort{fwm} puede contribuir a mejorar la calidad de la señal, reducir la tasa de errores de bit y aumentar la distancia efectiva de transmisión.


Teniendo en cuenta en cuenta lo mencionado previamente, en el presente trabajo de investigación se realizará un análisis comparativo del efecto de la inclusión de un algoritmo basado en un mecanismo dinámico para contrarrestar los efectos de la \gls{FWM}, dentro de una arquitectura de red con enfoque en la asignación de canales desigualmente espaciados.  A continuación, se describe el contenido de este trabajo de grado, expuesto en cuatro capítulos con base en la información obtenida en el desarrollo de la investigación. 
\vfill


\noindent{\textbf{Capítulo 1: Generalidades}}

En este capítulo se describen algunos aspectos generales sobre los sistemas de telecomunicaciones basados en fibra óptica. 
\vfill

\noindent{\textbf{Capítulo 2: Marco Metodológico}}

En este capítulo se definen la metodología y las herramientas de simulación 
\vfill

\noindent{\textbf{Capítulo 3: Diseño de la Arquitectura de Red}}

En este capítulo
\vfill

\noindent{\textbf{Capítulo 4: Diseño del Algoritmo}}

En este capítulo % \lipsum[2]
\vfill

\noindent{\textbf{Capítulo 5: Análisis del Desempeño de la Red Implementando el Algoritmo}}

En este capítulo se realiza el análisis del desempeño de los modelos 
\vfill

\noindent{\textbf{Capítulo 6: Conclusiones}}

En este capítulo se presentan las conclusiones, recomendaciones y trabajos futuros realizados 
\vfill


\noindent{\textbf{Palabras clave:}}

Mezcla de Cuadro Ondas (FWM), Multiplexación por División de Longitud de Onda, Monitoreo de Desempeño Óptico (OPM).
