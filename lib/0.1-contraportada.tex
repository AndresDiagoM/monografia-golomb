\newpage
\thispagestyle{empty}
\begin{center}

    {\textbf{\large Propuesta de un mecanismo dinámico que permita contrarrestar el efecto de FWM en una red MLR-PON }}\\
    \vfill

    Santiago Sánchez Echavarría\\
    100618011101 \\
    Andrés Felipe Diago Matta\\
    100618010965 \\
    \vfill

    Trabajo de Grado presentado a la Facultad de Ingeniería Electrónica y Telecomunicaciones de la Universidad del Cauca para optar por el título de Ingeniero en Electrónica y Telecomunicaciones\\
    \vfill
    
    Director: Phd. Jose Giovanny López Perafán \\
    Co-director: Phd. Carlos Andres Martos Ojeda \\
    \vfill

    \vfill
    Universidad del Cauca\\
    Facultad de Ingeniería en Electrónica y Telecomunicaciones\\
    Programa de Ingeniería en Electrónica y Telecomunicaciones\\
    Grupo de investigación de Nuevas Tecnologías en Telecomunicaciones - GNTT\\
    %Popayán, Cauca\\
    %Noviembre de 2023\\
    \date{\today}
    Popayán, \today
\end{center} 